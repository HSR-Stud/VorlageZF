% !TeX program = xelatex
% !TeX encoding = utf8
% !TeX root = Vorlage.tex

\documentclass[margin=normal]{hsrzf}

%%%%%%%%%%%%%%%%%%%%%%%%%%%%%%%%%%%%%%%%%%%%%%%%%%%
% Packages

%% The OST Student's package
\usepackage{oststud}

%% Language configuration
\usepackage{polyglossia}
\setdefaultlanguage[variant=swiss]{german}
\setotherlanguage{english}

%% License configuration
\usepackage[
  type={CC},
  modifier={by-nc-sa},
  version={4.0},
  lang={german},
]{doclicense}

%%%%%%%%%%%%%%%%%%%%%%%%%%%%%%%%%%%%%%%%%%%%%%%%%%%
% Metadata

\course{Elektrotechnik}
\module{ModAbk}
\semester{Fr\"uhlingssemester 2020}

\authoremail{vorname.name@ost.ch}
\author{\textsl{Vorname Name} -- \texttt{\theauthoremail}}

% did someone help you with this work?
\contributors{
  % I created this template, does that count?
  Naoki Pross
  % do not forget to add yourself!
}

\title{\texttt{\themodule} Zusammenfassung}
\date{\thesemester}

%%%%%%%%%%%%%%%%%%%%%%%%%%%%%%%%%%%%%%%%%%%%%%%%%%%
% Document

\begin{document}

% use roman numberals for introductiory pages
\pagenumbering{roman}

\maketitle

% \begin{abstract}
% \end{abstract}

% show the names of the people who contributed to this document.
% \section*{Contributors}
% \thecontributors

\section*{Lizenz}
\doclicenseThis

\tableofcontents

% actual content
\clearpage
\setcounter{page}{1}
\pagenumbering{arabic}

% I like to use 2 columns, you can remove this line to have one column.
\twocolumn

%%%%%%%%%%%%%%%%%%%%%%%%%%%%%%%%%%%%%%%%%%%%%%%%%%%
% Sample content

\begin{english}
  \section{How to use this template}

  \subsection{Introduction}

  This template is built on top of the
  \texttt{\textcolor{OSTBlackberry}{OST}stud} package, which provides many
  useful features such as sane defaults and nice macros (see
  \S\ref{sec:howto:sample}). The complete documentation can be found at

  \begin{center}
    \url{https://ctan.org/pkg/oststud}
  \end{center}

  By default the main language is German and the secondary language is English,
  you can change this on top of the document by editing the lines
  \lstinline|\setotherlanguage{}| and \lstinline|\setmainlanguage{}|. Further
  the default license is CC-BY-NC, also in German. This can also be changed by
  configuring the options of the \texttt{doclicense} package.

  \subsection{Examples} \label{sec:howto:sample}

  Here are some examples of the features provided by the
  \texttt{\textcolor{OSTBlackberry}{OST}stud} package.

  \subsubsection{Electrodynamics and Physics}

  The notation for vectors is ispired the notation of Prof. H.D. Lang. As we
  know Faraday's law in its integral form is
  \[
    \oint_{\partial A} \vec{E} \dotp d\vec{l}
      = -\frac{d}{dt} \int_A \vec{B} \dotp d\vec{s},
  \]
  however, anyone who has taken ELT3 knows that this expression can become very
  painful very fast. Thus, we consider only the harmonic solutions in the
  frequency domain by Fourier transforming this expression, i.e. we define
  $\underline{\vec{E}}(\omega) = \fourier\{\vec{E}(t)\}$ and similarly
  $\underline{\vec{B}}(\omega) = \fourier\{\vec{B}(t)\}$ using the convention
  such that
  \[
    \frac{d}{dt} \vec{E} 
    \quad\corresponds\quad
    j\omega \underline{\vec{E}}.
  \]

  This is particularly useful for solving the electromagnetic Wave equation
  \[
    \vlaplacian \vec{\tilde{E}} - \mu\sigma \partial_t \vec{\tilde{E}}
      - \mu\varepsilon \partial_t^2 \vec{\tilde{E}} = \vec{0},
  \]
  since the time derivatives become $j\omega$ and $\omega^2$. Then as you will
  learn for e.g. in the ElMag course $\underline{\vec{\tilde{H}}} = - \curl
  \underline{\vec{\tilde{E}}} / j\omega \mu_0$).

  \subsubsection{Mathematical Programming}

  Electrical engineers also work with a lot of mathematical programming and
  optimization problems, such as the linear quadratic regulator controller
  \[
    \vec{u}^\star = \argmin_{\vec{u}} \int_0^\infty
      \mt{\vec{x}} \mx{Q} \vec{x} + \mt{\vec{u}} \mx{R} \vec{u} \, dt,
  \]
  or Newton's method
  \[
    \vec{x}_{k+1} = \vec{x}_k + \minv{\mx{H}_f}(\vec{x}_k) \grad f(\vec{x}_k)
  \]
  where $f: \mathbb{R}^m \to \mathbb{R}$ and $\vec{x}_0$ is the initial guess.

  \subsubsection{Software and Code}

  We do a lot of maths, but sometimes we also write real code, either using
  really painful langauges such as C\texttt{++} or pseudocode
  (\textsc{Python}). For example see Fig. \ref{fig:cpp-code}.

  \begin{figure*}
    \begin{lstlisting}[language=c++]
#pragma once

#include <memory>
#include <set>

namespace flat {
  namespace core {
    enum class priority_t : unsigned  {
      max = 0,
      higher = 1,
      high = 2,
      none = 3,
      low = 4,
      lower = 5,
      min = 6,
    };

    class prioritized {
    public:
      const priority_t priority;

      prioritized(priority_t p = priority_t::none) : priority(p) {}
    };       

    struct prioritize {
      bool operator()(const prioritized& lhs, const prioritized& rhs) {
        return lhs.priority < rhs.priority;
      }

      bool operator()(const std::weak_ptr<prioritized> lhs,
                      const std::weak_ptr<prioritized> rhs) {

        if (auto l = lhs.lock()) {
          if (auto r = rhs.lock()) {
            // if both valid, check their priority
            // in case they are the same, left is prioritized
            return l->priority < r->priority;
          } else {
            // if right is expired, left is prioritized
            return true;
          }
        } else {
          // if left is expired, the right is prioritized
          return false;
        }
      }
    };

    template<typename Prioritized>
    using queue = std::multiset<Prioritized, prioritize>;
  }
} 
    \end{lstlisting}
    \caption{
      An abomination written in modern C\texttt{++}.
      \label{fig:cpp-code}
    }
  \end{figure*}

  \subsection{Example of documents}

  On HSR-Stud's GitHub organization you can find more documents written using
  (possibly an older version of) this template. For example:
  \begin{itemize}
    \item \url{https://github.com/HSR-Stud/RegT4}
    \item \url{https://github.com/HSR-Stud/FuVar}
    \item \url{https://github.com/HSR-Stud/DigDes_2021}
  \end{itemize}
  Feel free to go there and improve them if you find missing sections or
  errors.

\end{english}

% End of sample content
%%%%%%%%%%%%%%%%%%%%%%%%%%%%%%%%%%%%%%%%%%%%%%%%%%%

\end{document}
